\documentclass[a4paper, 10pt]{article}
\usepackage[utf8]{inputenc}
\usepackage[danish]{babel}
\usepackage{verbatim}

\title{Fundations of Computer Graphics\\ Ugeopgave 1}
\author{Mads Ohm Larsen}

\begin{document}
\maketitle

\section{Installtion af \texttt{Framework}}
Denne første del gøres meget kort.
Installation på en Linux maskine (Ubuntu Karmic Koala) foregik meget smertefrit, ved at downloade den givne zip-fil og pakke den ud.
Herefter køres\\ 

\noindent \texttt{ohm@amanda:\textasciitilde/projects/grafik\$ cmake GraphicsProject/}\\

\noindent Denne viser hvad den laver, her generere den \texttt{make}-filen og tjekker at alt er som det skal være.
Vi kan nu kører\\

\noindent \texttt{ohm@amanda:\textasciitilde/projects/grafik\$ make clean all}\\

\noindent Som har følgende output:
\begin{verbatim}
Scanning dependencies of target framework
[100%] Building CXX object CMakeFiles/framework.dir/src/main.o
Linking CXX executable framework
[100%] Built target framework
\end{verbatim}

Vi har nu installeret \texttt{Framework}, og kan kører dette med \texttt{./framework}.

\section{"Code-digging"}
I dette afsnit vil jeg kort forklare hvordan \texttt{drawPoint} funktionen fra \texttt{main.cpp} fungerere.
Formålet med dette er at komme tættere på koden og få et indblik i hvordan dette \texttt{framework} hænger sammen.

\texttt{drawPoint} funktionen løber følgende igennem:

\begin{enumerate}
\item Lader \texttt{render\_pipeline} indlæse \texttt{point\_rasterizer}
\item Sætter værdier for en firkant
    \begin{itemize}
    \item Her bliver sat en \texttt{x\_start} og en \texttt{x\_width}, som tilsamme giver en \texttt{x\_stop}. 
    \item Dette er en afgrænsning af en linie, og når vi gør dette med \texttt{y} også, er det en kasse. 
    \item I vores tilfælde bliver denne sat til at firkanten starter i $(150, 200)$ og slutter i $(550, 600)$\footnote{Øverste venstre hjørne er $(0,0)$}
    \end{itemize}
\item Der bliver sat en dybde (her $0.5$)
\item Gennemløber alle $(x,y)$ værdier i et dobbelt for-loop og laver en \texttt{vector3\_type} for hvert par.
\item Kalder \texttt{draw\_point} metoden hos \texttt{render\_pipeline} med farven gul
\item Det samme bliver gjort igen, denne gang for en firkant i $(250, 100)$ til $(750, 500)$ og i rød.
\end{enumerate}

Det gik jo nemt at få tegnet to firkanter, men hvad er det egenligt at \texttt{draw\_point} metoden hos rasterizeren gør?
Denne metode tager imod alle punkterne, samt farverne, der bliver tegnet.
Inde i selve \texttt{draw\_point} metoden hos pipelinen sker følgende:

\begin{enumerate}
\item Tjekker at vi har et \texttt{vertex\_program}, en \texttt{rasterizer} og et \texttt{fragment\_program}
\item Opretter output vektorer
\item Kalder \texttt{run} metoden hos \texttt{vertex\_program}. Denne tager imod den nuværende \texttt{state} i pipelinen, samt in- og output vektorer for koordinater og farver
\item Den kalder herefter \texttt{init}, hvis eneste opgave er at sætte nogle variable og sørger for at nulnormalen (\texttt{zero\_normal}) er nul. Den sætter også \texttt{valid} til sand, hvilket fortæller de andre metoder at \texttt{rasterizeren} er blevet initializeret
\item Så løber den alle fragmenter igennem, ved at spørge om der er \texttt{more\_fragments}
    \begin{itemize}
    \item Er der flere fragmenter, ser den på hvor de skal sidde (koordinater)
    \item Herefter tjekker den om vi overhovedet kan se dette fragment. Dette gøres med \texttt{ztest}
    \item Hvis vi kan se den, spørger vi \texttt{fragment\_program} hvilken farve den skal have
    \item Herefter indlæser vi den i z-bufferen, således at denne måske kan skygge for andre
    \item Og til sidst sender vi den til framebufferen med metoden hos pipelinen \texttt{write\_pixel\_to\_frame\_buffer}
    \end{itemize}
\end{enumerate}

Vi mangler nu kun at kigge på hvordan framebufferen tegner pixelen.
For at kigge på det dette, skal vi lidt længere tilbage, end hvor vi startede.
Vi skal kigge på hvornår \texttt{drawPoint()} bliver kaldt.
Dette sker når en person trykker på "p" på hans/hendes tastatur.
Når dette sker kalder vi nemlig både funktionen \texttt{drawPoint()}, men vi kalder også \texttt{flush()} metoden hos pipelinen og det er denne der sørger for at alle de pixels vi har samlet til framebufferen, bliver skrevet ud på skærmen.

\end{document}
