\documentclass[a4paper, 10pt]{article}
\usepackage[utf8]{inputenc}
\usepackage[danish]{babel}

\title{Fundations of Computer Graphics\\ Ugeopgave 1}
\author{Mads Ohm Larsen}

\begin{document}
\maketitle

\section{Installtion af \texttt{Framework}}
Denne første del gøres meget kort.
Installation på en Linux maskine (Ubuntu Karmic Koala) foregik meget smertefrit, ved at downloade den givne zip-fil og pakke den ud.
Her efter køres\\ 

\noindent \texttt{ohm@amanda:\textasciitilde/projects/grafik\$ cmake GraphicsProject/}\\

\noindent Denne viser hvad den laver, her generere den \texttt{make}-filen og tjekker at alt er som det skal være.
Vi kan nu kører\\

\noindent \texttt{ohm@amanda:\textasciitilde/projects/grafik\$ make clean all}\\

\noindent Som har følgende output:
\begin{verbatim}
Scanning dependencies of target framework
[100%] Building CXX object CMakeFiles/framework.dir/src/main.o
Linking CXX executable framework
[100%] Built target framework
\end{verbatim}

Vi har nu installeret \texttt{Framework}, og kan kører dette med \texttt{./framework}.

\section{"Code-digging"}

\end{document}
